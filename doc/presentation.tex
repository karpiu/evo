\documentclass{beamer}
\usetheme{Copenhagen}

\usepackage[utf8]{inputenc}
\usepackage[T1]{fontenc}
\usepackage[polish]{babel}
\usepackage{hyperref}
\usepackage{rotating}
\usepackage{listings}

\title{Zastosowanie algorytmów ewolucyjnych do zagadnień kombinatorycznych na przykładzie problemu szeregowania zadań}
\author{Michał Karpiński, Maciej Pacut}
\date{Grudzień 2011}

\begin{document}

\maketitle

\frame{\frametitle{Problem FlowShop}
\begin{itemize}
  \item Wykonujemy określoną liczbę zadań na określonej liczbie maszyn
\end{itemize} 
}

\frame{\frametitle{Dane wejściowe}
\begin{itemize}
  \item $n$ zadań \pause 
  \item $m$ maszyn \pause 
  \item macierz $T$, gdzie $T[i][j]$ jest czasem wykonania zadania $i$ na maszynie $j$ $(i = 1..n, j = 1..m )$
\end{itemize} 
}

\frame{\frametitle{Założenia}
\begin{itemize}
  \item Każde zadanie musi być wykonane co najwyżej raz na maszynach $1, 2 ... m$ (w tej kolejności) \pause
  \item Każda maszyna w danym momencie pracuje tylko nad jednym zadaniem \pause
  \item Każde zadanie w danym momencie jest wykonywane na co najwyżej jednej maszynie \pause
  \item Wykonywanie zadania nigdy nie jest zakłócone (przerwane) \pause
  \item Czas pomiędzy przekazaniem zadania z jednej maszyny na drugą jest zerowy
\end{itemize}
}

\begin{frame}{Algorytm Ewolucyjny - SGA}
  \begin{enumerate}
  \item Losuj populację początkową
  \item while(NOT Warunek Zakończenia)
  \item \{
  \item . . Krzyżowanie w obrębie populacji
  \item . . Mutacja osobników w populacji
  \item . . Zastąpienie populacji nową
  \item \}
  \end{enumerate}
\end{frame}

\begin{frame}{Operatory ewolucyjne}
  \begin{itemize}
    \item mutacja - losowa transpozycja z pdp $\frac{1}{3}$ \pause
    \item krzyżowanie - operatory PMX, OX, CX z pdp proporcjonalnym do funkcji przystosowania
  \end{itemize}
\end{frame}

\begin{frame}{Zestawy testowe}
  \begin{tabular}{ | c | c | c | }
      \hline                       
      Nazwa & N & M \\ \hline
      ta001-010 & 20 & 5 \\
      ta011-020 & 20 & 10 \\
      ta021-030 & 20 & 20 \\
      ta031-040 & 50 & 5 \\
      ta041-050 & 50 & 10 \\
      ta051-060 & 50 & 20 \\
      ta061-070 & 100 & 5 \\
      ta071-080 & 100 & 10 \\
      ta081-090 & 100 & 20 \\
      ta091-100 & 200 & 10 \\
      \hline
\end{tabular}
\end{frame}

\begin{frame}{Szukanie optimum}
  \begin{tabular}{l|l|l|l|l|l|} \cline{2-6}
     & \multicolumn{5}{|c|}{Liczba iteracji} \\ \cline{2-6}
                                & $10^4$  & $5\cdot 10^4$ & $10^5$ & $5\cdot 10^5$ & $10^6$ \\ \hline
    \multicolumn{1}{|c|}{t001} 	& $1.48$  & $1.25$        & $1.25$ & $0.31$        & $0.00$ \\ \hline
    \multicolumn{1}{|c|}{t005} 	& $1.21$  & $1.21$        & $1.21$ & $1.05$        & $1.05$ \\ \hline
    \multicolumn{1}{|c|}{t011} 	& $2.84$  & $0.69$        & $0.69$ & $0.69$        & $0.69$ \\ \hline
    \multicolumn{1}{|c|}{t015} 	& $2.81$  & $2.81$        & $2.81$ & $2.39$        & $0.21$ \\ \hline
    \multicolumn{1}{|c|}{t021} 	& $2.39$  & $1.56$        & $1.56$ & $1.56$        & $1.56$ \\ \hline
    \multicolumn{1}{|c|}{t025} 	& $2.18$  & $2.18$        & $2.18$ & $2.18$        & $2.18$ \\ \hline
    \multicolumn{1}{|c|}{t031} 	& $0.18$  & $0.18$        & $0.18$ & $0.18$        & $0.00$ \\ \hline
    \multicolumn{1}{|c|}{t035} 	& $0.03$  & $0.03$        & $0.03$ & $0.03$        & $0.03$ \\ \hline
    \multicolumn{1}{|c|}{t040} 	& $0.00$  & $-$           & $-$    & $-$           & $-$    \\ \hline
    \multicolumn{1}{|c|}{t061} 	& $0.00$  & $-$           & $-$    & $-$           & $-$    \\ \hline
    \multicolumn{1}{|c|}{t065} 	& $0.09$  & $0.00$        & $-$    & $-$           & $-$    \\ \hline
    \multicolumn{1}{|c|}{t080} 	& $0.99$  & $0.99$        & $0.99$ & $0.99$        & $0.56$ \\ \hline
    \multicolumn{1}{|c|}{t091} 	& $1.38$  & $0.81$        & $0.81$ & $0.81$        & $0.21$ \\ \hline  
    \multicolumn{1}{|c|}{t100} 	& $1.82$  & $0.69$        & $0.48$ & $0.48$        & $0.48$ \\ \hline
    \end{tabular}
\end{frame}

\begin{frame}{Porównanie algorytmu SGA do losowego przeszukiwania przestrzeni}
  \includegraphics[scale=1.5]{001b.pdf}
\end{frame}

\begin{frame}{Porównanie algorytmu SGA do losowego przeszukiwania przestrzeni}
  \includegraphics[scale=1.5]{018b.pdf}
\end{frame}

\begin{frame}{Porównanie algorytmu SGA do losowego przeszukiwania przestrzeni}
  \includegraphics[scale=1.5]{091b.pdf}
\end{frame}

\begin{frame}{Porównanie algorytmu SGA do losowego przeszukiwania przestrzeni}
  \includegraphics[scale=1.5]{100b.pdf}
\end{frame}

\begin{frame}{Wykresy średnich wartości populacji w zależności od
    iteracji; porównanie 3 operatorów: CX, OX, PMX}
   \includegraphics[scale=1.5]{003c.pdf}
\end{frame}

\begin{frame}{Wykresy średnich wartości populacji w zależności od
    iteracji; porównanie 3 operatorów: CX, OX, PMX}
   \includegraphics[scale=1.5]{040c.pdf}
\end{frame}

\begin{frame}{Wykresy średnich wartości populacji w zależności od
    iteracji; porównanie 3 operatorów: CX, OX, PMX}
   \includegraphics[scale=1.5]{091c.pdf}
\end{frame}

\begin{frame}{Wykresy średnich wartości populacji w zależności od
    iteracji; porównanie 3 operatorów: CX, OX, PMX}
   \includegraphics[scale=1.5]{093c.pdf}
\end{frame}

\begin{frame}{Wykresy średnich wartości populacji w zależności od
    iteracji; porównanie 3 operatorów: CX, OX, PMX}
   \includegraphics[scale=1.5]{100c.pdf}
\end{frame}

\begin{frame}{Porównanie 3 operatorów: CX, OX, PMX; liczba udanych krzyżowań}
  \includegraphics[scale=1.5]{010h.pdf}
\end{frame}

\begin{frame}{Porównanie 3 operatorów: CX, OX, PMX; liczba udanych krzyżowań}
  \includegraphics[scale=1.5]{097h.pdf}
\end{frame}

\begin{frame}{Porównanie 3 operatorów: CX, OX, PMX; liczba udanych krzyżowań}
  \includegraphics[scale=1.5]{098h.pdf}
\end{frame}

\begin{frame}{Porównanie 3 operatorów: CX, OX, PMX; liczba udanych krzyżowań}
  \includegraphics[scale=1.5]{100h.pdf}
\end{frame}

\begin{frame}{Ile ewaluacji wykona algorytm aby dojść do 5\% odchylenia od optimum}
  \includegraphics[scale=1.5]{096p.pdf}
\end{frame}

\begin{frame}{Ile ewaluacji wykona algorytm aby dojść do 5\% odchylenia od optimum}
  \includegraphics[scale=1.5]{097p.pdf}
\end{frame}

\begin{frame}{Ile ewaluacji wykona algorytm aby dojść do 5\% odchylenia od optimum}
  \includegraphics[scale=1.5]{098p.pdf}
\end{frame}

\begin{frame}{Ile ewaluacji wykona algorytm aby dojść do 5\% odchylenia od optimum}
  \includegraphics[scale=1.5]{100p.pdf}
\end{frame}

\begin{frame}{Wariancja populacji}
  \includegraphics[scale=1.5]{012v.pdf}
\end{frame}

\begin{frame}{Wariancja populacji}
  \includegraphics[scale=1.5]{013v.pdf}
\end{frame}

\begin{frame}{Wariancja populacji}
  \includegraphics[scale=1.5]{015v.pdf}
\end{frame}

\end{document}
