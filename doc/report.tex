\documentclass[12pt]{article}
\usepackage[utf8]{inputenc}
\usepackage[T1]{fontenc}
\usepackage[polish]{babel}
\title{Zastosowanie algorytmów ewolucyjnych do zagadnień
  kombinatorycznych na przykładzie problemu szeregowania zadań - sprawozdanie}
\author{Michał Karpiński, Maciej Pacut}
\date{Grudzień 2011}
\begin{document}
  \maketitle
\section{Wstęp}
Problem szeregowania zadań ma kilka różnych wariancji. W niniejszej pracy zajmujemy się problemem typu {\em Flow Shop}.
Problem ten jest problemem planowania produkcji, w którym $n$ zadań musi zostać wykonanych na $m$ maszynach (jeden po drugim).
Danymi wejściowymi są $n$, $m$ oraz macierz $T$, gdzie $T[i][j]$ jest czasem wykonania zadania $i$ na maszynie $j$ $(i = 1..n, j = 1..m )$.
Czasy są nieujemne, stałe i nie zmianiają się podczas pracy maszyn. Problem jest zminimalizowanie czasu 
pomiędzy rozpoczęciem wykonywania pierwszego zadania na pierwszej maszynie a zakończeniem wykonywania ostatniego zadania na ostatniej maszynie.
Dla naszego problemu przyjmujemy nastepujące założenia:

\begin{itemize}
  \item Każde zadanie musi być wykonane co najwyżej raz na maszynach $1, 2 ... m$ (w tej kolejności)
  \item Każda maszyna w danym momencie pracuje tylko nad jendym zadaniem
  \item Każde zadanie w danym momencie jest wykonywane na co najwyżej jednej maszynie
  \item Wykonaywanie zadania nigdy nie jest zakłócone (przerwane)
  \item Czas pomiędzy przekazaniem zadania z jednej maszyny na drugą jest zerowy
\end{itemize}

Tak zdefiniowany problem jest NP-trundy. W niniejszej pracy prezentujemy algorytm ewolucyjny znajdujący przybliżone rozwiązanie.

\section{Schemat algorytmu ewolucyjnego}
SGA. Funkcja przystosowania. Wybór osobników do następnej populacji.
Warunki zakończenia. Prawdopodobieństwo mutacji. Wielkość populacji.
\section{Funkcja celu}
Opis cmaxa
\section{Operatory krzyżowania}
PMX, CX, OX. Wspomnienie o stałej długości s,r.
\section{Operator mutacji}
Cały wielki rodział o jednej transpozycji!
\section{Przeprowadzone testy}
\subsection{Wariancja}
O ``odległości edycyjnej''.
\subsection{Test szybkości zbieżności do optimum}
W porównaniu do losowego przeszukiwania przestrzeni
\subsection{Które operatory krzyżowania psują}
\end{document}
